\section{Модел на излъчващата среда}
\lfoot{}

Представените оригинални резултати в глава {\color{red}X} се базират на цялостен модел на излъчващата среда около свръхмасивни компактни обекти. Това включва аналитични изрази за разпределението на материята $\rho(\vec{r}\,)$, профил на скоростта ѝ $u^\mu (x^\mu)$, температурния профил $T(\vec{r\,})$ и разпределението на магнитното поле $\vec{B}(\vec{r}\,)$. Формулирането на такъв модел изисква съгласуването на теоретични (в общия случай числени) разглеждания и наблюдателни данни, и представлява итеративен процес на надграждане на феноменологичните разбирания за процесите.\\

Най-прецизните измервания на физичната среда в околност на тези обект идва от интерферометричните измервания на колаборацията EHT и ALMA. С тяхна помощ може да се даде оценка за средната стойност на плътността, температурата и магнитното в излъчващата среда. Съпоставка на тези стойности с набор числени магнитохидродинамични симулации може да ни даде представа за динамичните процеси в зоната на акреция. Така подбраните числени модели могат да бъдат усредени във времето и от тях да бъдат извлечени аналитични функции за равновесното състояния на излъчващата зона. В тази глава ще обобщим основните резултати от наблюденията на колаборацията EHT, и ще представим използваните в литературата феноменологични модели на излъчващата среда, които са най-добре съгласувани с наблюденията. 


\subsection{Резултати от наблюденията на EHT}

