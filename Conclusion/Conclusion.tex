\section{Заключение}

В този дисертационен труд разгледахме обстойно наблюдателната проява на два големи класа екзотични компактни обекти, които не притежават хоризонт на събитията - пространствено-времевите тунели и голите сингуларности. Целяхме да отговорим на следният въпрос:\\

\emph{Възможно ли е различаването на екзотични компактни обекти, които не притежават хоризонт на събитията, от черни дупки в ОТО, чрез съществуващите и бъдещи наблюдения на колаборацията EHT.}\\

Започнахме с разглеждане на общите оптически свойства на тези компактни обекти. В глава 6 използвахме известният полу-аналитичен подход \cite{Muller2009}\cite{Gyulchev2020}\cite{Gyulchev2021}\cite{Deliyski2022} за генериране на образите на единични орбити. С това показахме, че и двата класа екзотични компактни обекти притежават съществено различна морфология на образите си, спрямо тази на черни дупки на Шварцшилд. Формира се централна пръстеновидна структура, разположена там, където би била сянката на обекта. Тази структура съответства (в случая на тунели) на фотони, преминали през гърловината и (в случая на голи сингулярности) на такива, които са се разсели от сингулярността. Наблюдателното засичане на подобни структури може да служи като ясен и еднозначен белег за отклонения от ОТО. \\

Мотивирани от това, и наскоро излезлите резултати за линейната поляризация, засечена от колаборацията EHT \cite{EHT_M87_VII}\cite{EHT_M87_VIII}, се насочихме към изследване на отпечатъкът на пространство-времето върху поляризацията, засечена от далечен наблюдател. За целта в глава 7 приложихме опростен аналитичен модел на излъчването \cite{Narayan2021}, обобщен за произволни статични и сферично симетрични метрики \cite{Delijski2022}\cite{Deliyski2023}. Моделът изисква пресмятането на вълновият вектор на фотоните върху източникът, което извършваме с разработеният от нас код Mjølnir. Използвайки този модел направихме следните заключения и за двата типа екзотични компактни обекта, които изследвахме:\\

\emph{1) Директните образи на излъчващата среда се влияят слабо от природата на пространство времето. Доминантният физически фактор, определящ техните свойства, е магнитното поле.}\\

\emph{2) Индиректните образи се влияят силно, както от магнитното поле, така и от природата на пространство-времето.}\\

Следователно за да наложим ограничения върху пространство-времето на база поляризацията на наблюдаваното лъчение, е нужно поне оптическото разделяне на образите с $n = 0$ и $n = 1$ от останалите.\\

Имайки този резултат в предвид, обърнахме внимание в \cite{Deliyski2024} (глава 8 от дисертацията) върху възможността за наблюдаване на образите с $n = 1$, както и екзотичните образи, представени в \cite{Deliyski2022}\cite{Gyulchev2020}\cite{Gyulchev2021}. За целта използваме няколко софтуерни пакета:\\

\textbf{1)} Оригиналният код Mjølnir. В него сме имплементирали цялостен модел на излъчващата среда и го използваме за генериране на "идеални" (т.е. с безкрайна разделителна способност) наблюдения на екзотични компактни обекти (виж допълнение Б).\\

\textbf{2)} Пакетът ehtim$^{19}$ \cite{EHTIM}, който има две основни функционалности: симулира реалистично наблюдение на образите, генерирани с помощта на Mjølnir, и също така извършва тяхната реконструкция, използвайки методологията на колаборацията EHT. С негова помощ получаваме образите на голи сингуларности за три различни набора на радио телескопи, и две различни наблюдателни честоти.\\

\textbf{3)} Пакетът VIDA$^{21}$ \cite{VIDA}, който фитира геометрични модели към подадените му изображения. Използваме го за да моделираме образите, получени от ehtim, като елипса с гаусова дебелина. С негова помощ дефинираме геометричните характеристики на реконструираните от ehtim образи, на които базираме заключенията си.\\

Показваме, че дори и с разширяване на набора телескопи, спрямо кампанията на EHT от 2017та година, наблюдения при честотата $230$ GHz \emph{не} могат да разделят нито класическите индиректни образи, нито екзотичната централна пръстеновидна структура. Образите остават морфологично сходни на черни дупки (т.е. пръстен с ясно изразена централна депресия и асиметрия на излъчването). Наблюдаваме обаче, че централната депресия показва по-висок поток. За оценка на този поток въвеждаме величината $\hat{f}_c$, представляваща отношението между минималният поток в депресията, и средният такъв на пръстеновидната структура. Получаваме, че той се различава с порядък между черни дупки и голи сингулярности, за конфигурацията на телескопи от 2017та година, като тази разлика расте до два порядъка, когато се добавят повече телескопи към наблюденията.\\

По-значимият резултат е, че ако увеличим честотата на наблюдаване до планирана за ngEHT 345 GHz, реконструкциите стават чувствителни към централната пръстеновидна структура. Наблюдаваме появата на ясен централен максимум в депресията, чийто относителен интензитет може да достигне 30\% от максимума на целият образ. Това е характерен белег, чиято поява в реални наблюдения може да служи като \emph{директно} експериментално потвърждаване на силно отклонение от ОТО, и служи като главното предсказание на този дисертационен труд.