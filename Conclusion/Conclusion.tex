\section{Заключение и обзор на научният принос}

В този дисертационен труд разгледахме обстойно наблюдателната проява на два големи класа екзотични компактни обекти, които не притежават хоризонт на събитията - пространствено-времевите тунели и голите сингуларности. Целяхме да отговорим на следният въпрос:\\

\emph{Възможно ли е различаването на екзотични компактни обекти, които не притежават хоризонт на събитията, от черни дупки, чрез съществуващите и бъдещи наблюдения на колаборацията EHT.}\\

Започнахме с разглеждане на общите оптически свойства на тези компактни обекти. В публикация I (и глава 5 от дисертацията) разширяваме съществуващи вече изследвания за голи сингулярности \cite{Gyulchev2020} и \cite{Gyulchev2021}, към пространствено-времеви тунели. Използвахме известният полу-аналитичен подход \cite{Muller2009} за генериране на образите на единични орбити. С това показахме, че и двата класа екзотични компактни обекти притежават съществено различна морфология на образите си, спрямо тази на черни дупки на Шварцшилд. Формира се централна пръстеновидна структура, разположена там, където би била сянката на обекта. Тази структура съответства (в случая на тунели) на фотони, преминали през гърловината и (в случая на голи сингулярности) на такива, които са се разсели от сингулярността. Наблюдателното засичане на подобни структури може да служи като ясен белег за съществуването на подобни обекти.\\

Мотивирани от това и наскоро публикуваните резултати за линейната поляризация, засечена от колаборацията EHT \cite{EHT_M87_VII} и \cite{EHT_M87_VIII}, се насочихме към изследване на отпечатъкът на пространство-времето върху поляризацията, засечена от далечен наблюдател. За целта в публикации II и III (глава \emph{6} от дисертацията) приложихме опростен аналитичен модел на излъчването \cite{Narayan2021}, обобщен за произволни статични и сферично симетрични метрики. Моделът изисква пресмятането на вълновият вектор на фотоните върху източникът, което извършваме с разработеният от авторът код Mjølnir. Използвайки този модел направихме следните заключения и за двата класа екзотични компактни обекта:\\

\emph{1) Директните образи на излъчващата среда се влияят слабо от природата на пространство времето. Доминантният физически фактор, определящ тяхното свойство е магнитното поле.}\\

\emph{2) Индиректните образи се влияят силно както от магнитното поле, така и от природата на пространство-времето. В зависимост от геометрията на магнитното поле, относителните отклонения на интензитета, спрямо черни дупки на Шварцшилд, може да достигне до порядък.}\\

Имайки този резултат в предвид, обърнахме внимание в публикация IV (глава \emph{7} от дисертацията) върху възможността за наблюдаване на тези, както и на екзотичните образи. За целта използваме няколко софтуерни пакета:\\

\textbf{1)} Авторският код Mjølnir. В него сме имплементирали цялостен модел на излъчващата среда и го използваме за генериране на "идеални" (т.е. с безкрайна разделителна способност) наблюдения на екзотични компактни обекти (виж допълнение \emph{Б} от дисертацията).\\

\textbf{2)} Пакетът ehtim \cite{EHTIM}, който има две основни функционалности: симулира реалистично наблюдение на образите, генерирани с помощта на Mjølnir, и също така извършва тяхната реконструкция, използвайки методологията на колаборацията EHT. С негова помощ получаваме образите на голи сингуларности за три различни набора на радио телескопи, и две наблюдателни честоти.\\

\textbf{3)} Пакетът VIDА \cite{VIDA}, който фитира геометрични модели към подадените му изображения. Използваме го за да моделираме образите, получени от ehtim, като елипса с гаусова дебелина. С негова помощ дефинираме геометричните характеристики на реконструираните от ehtim образи, на които базираме заключенията си.\\

Показваме, че дори и с разширяване на набора телескопи, спрямо кампанията на EHT от 2017та година, наблюдения при честотата $230$ GHz \emph{не} могат да разделят нито класическите индиректни образи, нито екзотичната централна пръстеновидна структура. Образите остават морфологично сходни на черни дупки (т.е. пръстен с ясно изразена централна депресия и асиметрия на излъчването). Наблюдаваме обаче, че централната депресия показва по-висок поток. За оценка на този поток въвеждаме величината $\hat{f}_c$, представляваща отношението между минималният поток в депресията, и средният такъв на пръстеновидната структура. Получаваме, че той се различава с порядък между черни дупки и голи сингулярности, за конфигурацията на телескопи от 2017та година, като тази разлика расте до два порядъка когато се добавят повече телескопи към наблюденията.\\

По-значимият резултат е, че ако увеличим честотата на наблюдаване до планирана за ngEHT 345 GHz, реконструкциите стават чувствителни към централната пръстеновидна структура. Наблюдаваме появата на ясен централен максимум в депресията, чийто относителен интензитет може да достигне 30\% от максимума на целият образ. Това е характерен белег, чиято поява в реални наблюдения може да служи като признак за съществуването на екзотични компактни обекти.

\newpage

\section{Списък с научната активност}

Този дисертационен труд се бази на три приети публикации, и една (към дата на писане) в процес на рецензия. Те са номерирани с римски цифри (I, II, III и IV) в текста. Тук представяме техен списък.

\subsection{Списък с научни публикации}

$\bullet$ Публикация I - V Deliyski, G Gyulchev, P Nedkova, and Yazadjiev. Observational features of thin accretion disks around traversable wormholes. Journal of Physics: Conference Series, 2255(1):012002, apr 2022.\\

\noindent$\bullet$ Публикация II - Valentin Deliyski, Galin Gyulchev, Petya Nedkova, and Stoytcho Yazadjiev. Polarized image of equatorial emission in horizonless spacetimes. Traversable wormholes. Phys. Rev. D, 106:104024, Nov 2022.\\

\indent$\circ$ Тя също така е представена във Physics Synopsis - \\\indent https://physics.aps.org/articles/v15/s154\\

\noindent$\bullet$ Публикация III - Valentin Deliyski, Galin Gyulchev, Petya Nedkova, and Stoytcho Yazadjiev. Polarized image of equatorial emission in horizonless spacetimes: Naked singularities. Phys. Rev. D, 108:104049, Nov 2023.\\

\noindent$\bullet$ Публикация IV - Valentin Deliyski, Galin Gyulchev, Petya Nedkova, and Stoytcho Yazadjiev. Observing naked singularities by the present and next-generation event horizon telescope. http://arxiv.org/abs/2401.14092, 2024.

\subsection{Списък с изнесени доклади}

$\bullet$ Изнесен е доклад на „Национален форум за
съвременни космически изследвания 2021“, на тема „Наблюдателни
белези на свръхкомпактни обекти с акреционни дискове“ на 08.10.2021.\\

\noindent$\bullet$ Изнесен е доклад по време на научно посещение при Emmy Noether Research Group: Gravitational waves from compact objects към  университетът Eberhard Karl в Тюбинген, Германия на 07.03.2024.\\

\noindent$\bullet$ Изнесен доклад на Seventeenth Marcel Grossman Meeting на тема "Polarized image of equatorial emission in horizonless spacetimes".
