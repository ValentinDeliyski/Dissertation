\section{Наблюдения на екзотични компактни обекти}
Както видяхме в глава 6, екзотичните компактни обекти могат да генерират качествено различни (от черни дупки на Кер) образи на излъчващата си среда. Също в глава 7 показахме, че поляризацията на класическите образи с $n = 1$ се влияят силно от природата на пространство-времето. Естествено е да зададем въпроса, имайки предвид съвременните ни наблюдателни техники, възможно ли е експерименталното потвърждение на предсказанията от предишните две глави?\\

До момента на писане, единствените директни наблюдения на свръхмасивни компактни обекти, с разделителна способност, съизмерима с мащабите на релативистките образи, са осъществени от колаборацията EHT. Самата методика на наблюдение, и последствено реконструиране на изображение, са описани в допълнение ВЛБИ ДОПЪЛНЕНИЕ. Ние тук ще се фокусираме само върху крайният резултат на тяхната процедура - изображенията. Можем да забележим от обзора ни в глава 3, че наблюденията през 2017 г. нямат достатъчна разделителна способност за да отделят образите с $n > 0 $ от директните. Тогава можем да повдигнем следните въпроси: Възможно ли е наблюденията на EHT от 2017 г., или следващи такива:\\

\textbf{1)} Да засекат по \emph{еднозначен} начин наличието на екзотични образи на излъчващата среда?\\

\textbf{2)} Да отделят образите с $n = 1$ от всички останали?\\

За да отговорим на тези два въпроса ще е нужно да симулираме самите наблюдения. Колаборацията EHT е предоставила софтуерен пакет, наречен ehtim\footnote{EHTIM LINK}, който прави точно това. Той използва за вход "идеални наблюдения"\footnote{В този контекст е важно да разграничаваме между симулираните наблюдавани образи, и тези генерирани от решаване на уравненията на геодезичните (2.18) заедно с уравнението за лъчист пренос (2.32) и паралелен пренос на поляризацията (2.14). Първите ще наричаме \textbf{реконструирани} образи, докато вторите \textbf{идеални}.} на обектите, конфигурация от радио телескопи, и настройки на алгоритъма за реконструкция. \\

За генерирането на идеалните$\,$ образи на свръхкомпактни обекти, ние използваме феноменологичен модел на радиационно не-ефективна акреция, представен в част 8.1, с параметри, следващи от наблюдателната кампания на EHT от 2017 г. (виж глава 3). Самото получаване на тези образи, чрез решаване на системата (2.14), (2.18) и (2.32), извършваме с численият код Mjølnir. \\

Разглеждаме три различни конфигурации на радио телескопи, съответстващи на наблюдателните кампании през 2017 г., 2022 г. и плануваният ngEHT (виж ДОЛЪНЕНИЕ ВЛБИ).\\

Настройките на алгоритъма за реконструкция фиксираме на база на ЦИТЕ.\\

Като последна стъпка, характеризираме морфологията на реконструкциите с помощта на т.н. \emph{темплейтен анализ}. Той ни позволява по систематичен начин да дефинираме коя част на образа принадлежи към централната депресия, и коя към пръстеновидната структура, която видяхме в глава 3. За тази цел използваме софтуерният пакет VIDA\footnote{VIDA LINK}.

\subsection{Модел на излъчващата среда}

Ще разгледаме аналитичен модел, описващ геометрично и оптически тънък, радиационно неефективен (RIAF) акреционен диск в режим на магнитно заключване (MAD). Моделът е подбран така, че да е качествено сходен с GRMHD симулации ЦИТЕ. Следвайки ЦИТЕ приемаме, че излъчването е синхотронно, и се дължи на два отделни електронни ансамбъла - топлинен и не-топлинен. Описваме колективно и двата със степенен закон в радиалната посока, и Гаусов профил във вертикалната:

\begin{equation}
	n_e(r,z) = n_0\left(\frac{r}{r_0}\right)^{-2}e^{-\frac{z^2}{2(\alpha\rho)^2}}
	\begin{cases}
		e^{-\frac{(r-r_0)^2}{r^2_{\text{sc}}}},\quad 0 < r < r_0,\\
		1,\,\,\qquad\qquad r>r_0
	\end{cases}
\end{equation}
Тук параметърът $r_0$ определя положението на най-високо сгъстяване на диска, и заедно с експоненциалният множител при $0 < r < r_0$, служи за фиксиране на позицията на видимия излъчващ регион. Цилиндричните координати $\rho$ и $z$ се задават като $\rho = r\sin\theta$, $z = r\cos\theta$ и параметъра $\alpha$ определя ъгъла на отваряне на диска $\theta_{\text{op}}$ според $\alpha = \tan\theta_\text{op}$. За да генерираме тънък диск, ще фиксираме $\alpha = 0.1 \rightarrow \theta_{\text{op}}\approx 5.71^\circ$ за всички наши симулации.\\
В изследванията си целим да пресъздадем физичните условия при които наблюдаваме обекта M87$^*$, и затова ще подберем $r_0$ и $r_\text{sc}$ така, че да получим диаметър на видимия образ $d_\text{img}\approx 50\, \mu\text{arcsec}$.\\

Заедно с (8.1), също трябва да зададем и температурен профил (отново следвайки ЦИТЕ):

\begin{equation}
	T_e(r,z) = T_0\left(\frac{r}{r_0}\right)^{-1}
	\begin{cases}
		e^{-\frac{(r-r_0)^2}{r^2_{\text{sc}}}},\quad 0 < r < r_0,\\
		1,\,\,\qquad\qquad r>r_0
	\end{cases}
\end{equation}

Параметрите $n_0$ и $T_0$, които съответстват на екваториалните стойности на плътността и температурата при $r = r_\text{sc}$ определят стойността на наблюдавания поток. Понеже и двата параметъра са ограничени само от пълният поток, избираме да фиксираме $n_0$, и да варираме $T_0$ така, че да получим наблюдаван поток $F_{\text{230 GHz}} \approx 0.5 \text{Jy}$\\

Следващата стъпка в изграждането на модела е задаването на магнитното поле $\vec{B}$. Най-удобно е то да се зададе в собствената отправна система на флуида, където се задават и функциите на излъчване, които ще коментираме по-надолу. Избираме да работим в термини на параметъра на намагнитване $\sigma$:
\begin{equation}
	\sigma = \frac{B^2}{4\pi m_pc^2n_e},
\end{equation}
където $B$, $m_p$, $c$ и $n_e$ са съответно големината на магнитното поле, масата на протона, скоростта на светлината и концентрацията на електрони, пресметнати в \emph{Гаусовата система на единици}. За възпроизвеждане на наблюденията на EHT от 2017г. е достатъчно да приемем диска за \emph{равномерно намагнитен} - т.е. да фиксираме $\sigma = \text{const}$. Следвайки предишни разработки ЦИТЕ, избираме $\sigma = 0.01$. Това обаче само фиксира големината на полето. Синхотронното излъчване се влияе силно и от геометрията на това поле. За целите на това изследване обаче, точната геометрия на полето не е важна, и ние избираме да усредним по всички възможни такива. Това на практика се свежда до усредняване на функциите на излъчване по ъгъла $\alpha = \arccos\frac{\vec{k}\cdot\vec{B}}{|\vec{k}||\vec{B}|}$, където $\vec{k}$ е локалният 3-мерен вълнов вектор на фотона. Следователно геометрия на полето \emph{няма} да задаваме.\\

Самото излъчване от диска приемаме за синхотронно в свръх-релативистката граница (подсказано от високата температура, обсъдена в глава 3). Този механизъм на излъчване е подробно разгледан в допълнение А. За целите на конкретното изследване не се интересуваме от поляризацията на лъчението, а само от пълният му интензитет. Следователно в уравнението за лъчист пренос (2.32) можем да приемем за ненулеви само коефициентите $\{j_{I,\nu}, \kappa_{I,\nu}\}.$ Едно опростяващо приближение което ще приемем е, че всичките излъчващи електрони са разпределени по скорости топлинно. Това е приближение с което работи и екипа на EHT в анализа си ЦИТЕ. Те също показаха, че то се отразява главно върху оценката на темпа на акреция $\dot{M}$, който не е релевантен за това изследване. Следователно приемаме следните изрази за $\{j_{I,\nu}, \alpha_{I,\nu}\}$ (виж А. НЕЩО СИ):
\begin{subequations}
	\begin{equation}
		j_{I,\nu}\approx n_e \frac{\sqrt{2}\pi e^2\nu_s}{3cK_2(\Theta_e^{-1})}\left(X^{1/2} + 2^{11/12}X^{1/6}\right)^2 e^{-X^{1/3}}
	\end{equation}
	\begin{equation}
		\alpha_{I,\nu} = \frac{j_{I,\nu}}{B_\nu(T)}
	\end{equation}
\end{subequations}
Където $B_\nu(T)$ е функцията на Планк, $\Theta_e = k_BT/mc^2$ и $K_2$ e модифицирана функция на Бесел от втори род. Отделно сме дефинирали величините:
\begin{equation}
	X = \frac{\nu}{\nu_s},\quad \nu_s = \frac{2}{9}\nu_\text{cyclo}\Theta_e^2\sin\alpha, \quad \nu_\text{cyclo} = \frac{eB}{2\pi m c}.
\end{equation}
Важно е да отбележим, че величините, участващи в (8.4) и (8.5), са пресметнати в \emph{Гаусова система единици}. Тогава усредняването на (8.4а) се дава с:
\begin{equation}
	j_{I,\nu}\rightarrow\langle j_{I,\nu} \rangle = \frac{1}{4\pi}\int j_{I,\nu} d\Omega = \frac{1}{2}\int j_{I,\nu} \sin\theta d\theta.
\end{equation}
Последната стъпка в изграждането на модела е задаването на профил на скоростта на акреционният диск. Следваме ЦИТЕ и приемаме 4-скорост от вида:
\begin{equation}
	u_\mu dx^\mu = u_0(-dt + \ell d\phi),\quad \ell = \frac{\rho^{3/2}}{1 +\rho}
\end{equation}
Нормирането на $u_\mu$ фиксира стойността на $u_0$:
\begin{equation}
	u_0 = \frac{1}{\sqrt{-(g^{tt} - 2g^{t\phi}\ell + g^{\phi\phi}\ell^2)}}
\end{equation}
По време на извършване на симулациите обаче установихме, че формата на $\ell$ (8.7), води до 4-скорост която не е винаги добре дефинирана за всички разгледани метрики. Следователно въвеждаме корекциите:
\begin{equation}
	\ell\rightarrow\begin{cases}
		\ell \left(1 - \frac{2M}{\gamma r}\right)^{\gamma}, \quad\text{за решението на Джанис-Нюман-Уиникър}\\
		\ell \left(1 - \frac{b}{r}\right), \,\,\,\qquad\text{за пространствено-времеви тунели}.
	\end{cases}
\end{equation}
\subsection{Симулирани идеални образи на M87$^*$}

Както споменахме в увода, тези образи генерираме с помощта на численият код Mjølnir. Тъй като целта ни е да пресъздадем наблюденията на M87$^*$, избираме инклинация на наблюдателят $i = 160^\circ$, маса на компактният обект $M = 6.2\times 10^9M_\odot$ и разстояние до него $D = 16.9\, \text{Mpc}$ CITE. Пълен списък с параметрите на модела, общи за всички направени симулации е представен в таблица \ref{table:Common_ray_tracer_params}.\\

\begin{table}[h!]
	\centering
	\begin{tabular}{||c|c||}
		\hline
		\hline
		\thead{ Параметър }   &\thead{Стойност} \\
		\hline
		\thead{Маса на компактният обект $M$}  &  \thead{$6.2\times10^9M_\odot$}\\  
		\hline
		
		\thead{Разстояние до компактният обект} &  \thead{$16.9$ Mpc}\\
		\hline
		
		\thead{Ъгъл на отваряне на диска ($\alpha = \tan\theta_{\text{op}}$)}  & \thead{0.1}\\
		\hline
		
		\thead{Концентрация на електрони при $r = r_0,\,\theta = \frac{\pi}{2}$}  & \thead{5$\times10^2$cm$^{-3}$}\\
		\hline
		
		\thead{Намагнитеност на диска $\sigma$}  & \thead{0.01}\\
		\hline
		
		\thead{Параметър на "острота"$\,r_\text{sc}$} & \thead{0.4M}\\
		\hline
		
		\thead{Инклинация на наблюдателя $i$}  & \thead{160$^\circ$}\\
		\hline
		
		\thead{Резолюция} & \thead{$2048\times2048$}\\
		\hline
		
		\thead{Зрително поле} &  \thead{$100\times100\,\,\mu\text{arc}\sec$}\\
		\hline
		\hline
	\end{tabular}
	\caption[Общи параметри за всички ray-tracing симулации.]{Общи параметри за всички ray-tracing симулации.}
	\label{table:Common_ray_tracer_params}
\end{table}

Първо разглеждаме класически черни дупки за Кер, които ще използваме като "базовият"$\,$ случай за сравнение с екзотичните компактни обекти.

\newpage

\begin{figure}[h!]
	\centering
	\begin{subfigure}{12cm}
		\hspace{-1cm}
		\includegraphics[scale = 0.3]{Ray_tracer_plot_230_Sch.png}
	\end{subfigure}\\
	\begin{subfigure}{12cm}
		\hspace{-1cm}
		\includegraphics[scale = 0.3]{Ray_tracer_plot_230_Kerr_0.5.png}
	\end{subfigure}\\
	\label{Kerr_Ray_tracer_230}
	\caption[Идеални образи на черни дупки на Кер с реалистичен модел на излъчващата среда, при различен параметър на въртене $a$]{Идеални образи на черни дупки на Кер с реалистичен модел на излъчващата среда, при различен параметър на въртене $a = \{0, 0.5\}$. Температурата при $r = r_0,\,\,\theta = \frac{\pi}{2}$ за двете симулации е фиксирана на $T_0 = 6.8\times10^9$ K и $r_0 = 4.5M$. Пълният поток е $\mathcal{F}_{\text{tot}} = 0.574$ Jy за Шварцшилд, и $\mathcal{F}_{\text{tot}} = 0.582$ Jy за Кер. За останалите параметри виж таблица \ref{table:Common_ray_tracer_params}.} 
\end{figure}

Разглеждаме ефекта на въртенето като извършваме симулации при стойности на параметъра на въртене $a = \{0, 0.5\}$. Резултатите са показани на фигура 8.1, където също сме начертали сечение на яркостната температура през правата $\delta_{\text{rel}}$ на образите. Докато на фигура 8.2 са представени резултатите за гола сингулярност на Гаус-Боне и на Джанис-Нюман-Уиникър за избрани стойности на параметрите $\gamma$ в метриките им.\\

\noindent Виждаме, че интензитетът на централните образи е значителен. Той представлява максималният за целият образ на Джанис-Нюман-Уиникър, а за Гаус-Боне е само леко занижен, спрямо този на директният образ. Дори и оптическото разделяне на тези образи от EHT да е трудно, значителният поток от централните образи би се отразил върху реконструкцията на образите. В следващите глава оценяваме количествено този ефект.
\newpage

\begin{figure}[h!]
	\centering
	\begin{subfigure}{12cm}
		\hspace{-1cm}
		\includegraphics[scale = 0.3]{Ray_tracer_plot_230_JNW.png}
	\end{subfigure}\\
	\begin{subfigure}{12cm}
		\hspace{-1cm}
		\includegraphics[scale = 0.3]{Ray_tracer_plot_230_GB.png}
	\end{subfigure}\\
	\label{Naked_Singularity_Ray_tracer_230}
	\caption[Идеални образи на голи сингулярности с реалистичен модел на излъчващата среда, при избрани стойности на $\gamma$.]{Идеални образи на голи сингулярности с реалистичен модел на излъчващата среда, при избрани стойности на $\gamma$. Температурата при $r = r_0,\,\,\theta = \frac{\pi}{2}$ за Джанис-Нюман-Уиникър е $T_0 = 7.2\times10^9$ K, докато за Гаус-Боне е $T_0 = 5.9\times10^9$. Пълният поток е съответно $\mathcal{F}_{\text{tot}} = 0.574$ Jy и $\mathcal{F}_{\text{tot}} = 0.582$ Jy. Параметърът $r_0$ е фиксиран на $r_0 = 5M$ за двете решения. За останалите параметри виж таблица \ref{table:Common_ray_tracer_params}.} 
\end{figure}


\subsection{Реконструкция на образите}

Както вече споменахме, използваме библиотеката ehtim$^{15}$ за реконструцията на образите на компактните обекти от фигура 8.1 и 8.2. Методиката зад реконструциите е обсъдена в ДОПЪЛНЕНИЕ ВЛБИ. Тук само ще обобщим използваните настройки на ehtim - параметрите на симулираното наблюдение, и тези на алгоритъма за реконструкция. След това ще коментираме получените резултати.\\

Разглеждаме 3 конфигурации на телескопи - тази от кампанията от 2017 г., 2022г. и преспективна конфигурация за бъдещи наблюдения, наричана ngEHT. Първите две наблюдават единствено на честота $230$ GHz, докато ngEHT наблюдава на $230$ GHz и 345 GHz. Физическите параметри на самите телескопи са дадени в таблица НЕЩО СИ. \\

Първо използваме пакета ehtim за да генерираме т.н. \emph{синтетично наблюдение}. То представлява симулация на $(u,v)$ покритието при реално наблюдение и има следните входни параметри: време на интеграция $\Delta t$, време между интегрирания $T$, продължителност на наблюдението $T_\text{obs}$ и широчина на честотната лента $\Delta\nu$. Избраните от нас параметри са съобразени с ЦИТЕ, и са дадени в таблица \ref{table:ehtim_obs_settings}.\\

\begin{minipage}{18em}
	\begin{center}
		\begin{tabular}{|| m{7.5em} | m{5em} | m{2em} ||}
			\hline 
			Конфигурация от телескопи & \multicolumn{2}{m{7em}||}{Параметри на синтетичните наблюдения} \\
			\hline
			\multirow{4}{7.5em}{\centering \small EHT 2017 / 2022} &\centering $\Delta t,\, [s]$    		& 5   \\ 
														&\centering $T,\,[s]$ 		     		& 30  \\ 
														&\centering $T_\text{obs},\,[h]$ 		& 24  \\
														&\centering $\Delta \nu,\,[\text{GHz}]$ & 4 \\
			\hline
			\multirow{4}{7.5em}{\centering \small ngEHT} 		  & \centering $\Delta t,\, [s]$    	   & 120 \\ 
													  & \centering $T,\,[s]$ 		      	   & 600 \\ 
												      & \centering $T_\text{obs},\,[h]$ 	   & 24  \\
												      & \centering $\Delta \nu,\,[\text{GHz}]$ & 2 \\
			\hline
		\end{tabular}
	\end{center}
	\captionof{table}[Настройки на синтетичните наблюдения.]{Настройки на синтетичните наблюдения.}
	\label{table:ehtim_obs_settings}
\end{minipage}\,\,
\begin{minipage}{18em}
	За настройките на алгоритъма за реконструкция следваме ЦИТЕ. Избираме да работим с два члена $\chi^2(I,d)$ члена: $\chi^2_\text{amp}$ и $\chi^2_\text{cl. phase}$. Избираме също така четири регуляризатора: $S_\text{entropy}$, $S_\text{TSV}$, $S_\text{tot flux}$ и $S_\text{centroid}$. Стойностите на хиперпараметрите $\alpha_D$ и $\beta_R$, както и броя стадии и итерации на алгоритъма са обобщени в таблица \ref{table:reconstruction_settings}. С цел увеличаване на сходимостта на алгоритъма, правим конвулюция на полученото изображение след всеки стадии с Гаусов сигнал, имащ стандартно отклонение $\sigma = f_\text{blur} \sigma_{\text{230 GHz}}$, където $\sigma_{\text{230 GHz}}$ е номиналната резолюция на цялата конфигурация от телескопи при 230 GHz.
\end{minipage}

\begin{table}[h!]
	\centering
	\begin{tabular}{||c|c|c|c|c|c|c|c|c||}
		\hline
		\hline
		\thead{ Стадии } & \thead{$f_\text{blur}$} &\thead{$\beta_\text{entropy}$} &\thead{$\beta_\text{TSV}$} &\thead{$\beta_\text{tot flux}$} & $\beta_\text{centroid}$
						 & \thead{$\alpha_\text{amp}$} & \thead{$\alpha_{\text{cl. phase}}$} & $N_\text{iter}$\\
		\hline
		\thead{1}  &  \thead{NA} & \thead{1} &\thead{1} &\thead{100} & \thead{100} &\thead{100} &\thead{200} &\thead{1000} \\  
		\hline
		
		\thead{2}  &  \thead{0.75} & \thead{1} &\thead{50} &\thead{50} & \thead{50} &\thead{100} &\thead{75} &\thead{3000} \\  
		\hline
		
		\thead{3}  &  \thead{0.5} & \thead{1} &\thead{100} &\thead{10} & \thead{10} &\thead{100} &\thead{50} &\thead{4000} \\  
		\hline
		
		\thead{4}  &  \thead{0.33} & \thead{1} &\thead{500} &\thead{1} & \thead{1} &\thead{100} &\thead{100} &\thead{4000} \\  
		\hline
		\hline

	\end{tabular}
	\caption[Параметри на алгоритъма за реконструкция.]{Параметри на алгоритъма за реконструкция.}
	\label{table:reconstruction_settings}
\end{table}

Финалното изображение отново конвулираме с Гаусов сигнал, имащ $\sigma = \sigma_\text{clean} / 2$, където $\sigma_\text{clean}$ е стандартното отклонение на "чистия сноп". Това се прави, понеже в противен случай алгоритъма би произвел изображение с резолюция, много по-голяма от тази на самите телескопи.\\
\newpage
\subsubsection{Реконструкция от EHT 2017 / 2022}

На фигури 8.3 и 8.4 показваме реконструкциите на образите, получени от симулациите, показани на фигури 8.1 и 8.2. За всеки реконструиран образ даваме получените стойности за функциите $\chi^2_\text{amp}$ и $\chi^2_\text{cl. phase}$. Получаваме, че реконструкциите от двата набора телескопи EHT 2017 и EHT 2022 са изключително сходни и затова избираме да покажем само тези от EHT 2017, но количествения анализ в подточка 8.4 ще бъде представен и за двете конфигурации.


\begin{figure}[h!]
	\centering
	\begin{subfigure}{12cm}
		\hspace{-1.5cm}
		\includegraphics[scale = 0.23]{Ehtim_plot_2017_no_blur_Sch.png}
	\end{subfigure}\\
	\begin{subfigure}{12cm}
		\hspace{-1.5cm}
		\includegraphics[scale = 0.23]{Ehtim_plot_2017_no_blur_Kerr.png}
	\end{subfigure}\\
	\label{Kerr_EHT_2017}
	\caption[Реконструирани образи на черни дупки на Кер при различни параметри на въртене.]{Реконструирани образи на черни дупки на Кер при различни параметри на въртене $a = \{0, 0.5\}$. Левият панел показва "голата"$\,$ реконструкция, преди коволюцията с "чистия сноп". Финалните стойности на $\chi^2$ са $\chi^2_\text{amp} = \{1.02, 1.04\}$ и $\chi^2_\text{cl. phase} = \{0.9, 1.07\}$ за $a = \{0, 0.5\}$.} 
\end{figure}

Виждаме от фигура 8.4, че ефективната резолюция на набора телескопи не е достатъчно висока при $230$ GHz за да различи наличието на екзотичните образи. Те се "размиват"$\,$ и сливат с останалите. Забелязваме обаче, че това води до значително повишен поток в централната депресия. Можем да оценим количествено потока от този регион и да дефинираме с това мярка, по която да съдим за наличието на екзотични образи. 

\newpage
\begin{figure}[h!]
	\centering
	\begin{subfigure}{12cm}
		\hspace{-1.5cm}
		\includegraphics[scale = 0.23]{Ehtim_plot_2017_no_blur_JNW.png}
	\end{subfigure}\\
	\begin{subfigure}{12cm}
		\hspace{-1.5cm}
		\includegraphics[scale = 0.23]{Ehtim_plot_2017_no_blur_GB.png}
	\end{subfigure}\\
	\label{Naked_Singularity_EHT_2017}
	\caption[Реконструирани образи на голи сингулярности, при избрани стойности на $\gamma$, от EHT 2017]{Реконструирани образи на голи сингулярности, при избрани стойности на $\gamma$, от EHT 2017. Левият панел показва "голата"$\,$ реконструкция, преди коволюцията с "чистия сноп". Финалните стойности на $\chi^2$ са $\chi^2_\text{amp} = \{1.01, 1.00\}$ и $\chi^2_\text{cl. phase} = \{0.91, 0.84\}$ съответно за Гаус-Боне и Джанис-Нюман-Уиникър.} 
\end{figure}

\subsubsection{Реконструкция от ngEHT}

Преспективните бъдещи наблюдения на ngEHT освен, че ще включват повече телескопи (тук сме разгледали набор от 21 такива), също ще наблюдават на втора, по-висока честота $\nu = 345$ GHz. По-големият набор от телескопи би подобрил $(u,v)$ покритието, но наличието на втората честота се очаква значително да подобри ефективната резолюция на телескопа. На фигура АСФ са показани реконструкциите на образите (8.2). Виждаме въпреки, че морфология все още не е разделена, наблюденията на 345 GHz вече стават чувствителни към наличието на екзотичните образи. Появява се ясен локален максимум, намиращ се в централната депресия. В следващата подточка ще дадем количествена оценка за тази депресия.
\newpage
\begin{figure}[h!]
	\centering
	\begin{subfigure}{12cm}
		\hspace{-1.5cm}
		\includegraphics[scale = 0.23]{Ehtim_plot_ngEHT_no_blur_345_JNW.png}
	\end{subfigure}\\
	\begin{subfigure}{12cm}
		\hspace{-1.5cm}
		\includegraphics[scale = 0.23]{Ehtim_plot_ngEHT_no_blur_345_GB.png}
	\end{subfigure}\\
	\label{Naked_Singularity_EHT_ng2017}
	\caption[Реконструирани образи на голи сингулярности, при избрани стойности на $\gamma$, от ngEHT]{Реконструирани образи на голи сингулярности, при избрани стойности на $\gamma$, от ngEHT. Левият панел показва "голата"$\,$ реконструкция, преди коволюцията с "чистия сноп". Финалните стойности на $\chi^2$ са $\chi^2_\text{amp} = \{1.00, 0.99\}$ и $\chi^2_\text{cl. phase} = \{1.53, 1.46\}$ съответно за Гаус-Боне и Джанис-Нюман-Уиникър.} 
\end{figure}

\subsection{Темплейтен анализ}

За да направим количествено описание на реконструкциите, трябва да въведем величини, характеризиращи геометрията им. Колаборацията EHT въведоха за целта темплейт на пръстен с Гаусова дебелина, който фитираха към изображенията си ЦИТЕ. Той се характеризита с диаметър, дебелина и ориентация. Фитираните стойности на тези параметри се приемат за геометричните характеристики на реконструкциите. Ние ще подходим по подобен начин, възползвайки се от софтуерният пакет VIDA$^{17}$ CITE. Той приема за "вход"$\,$ реконструиран образ, към който фитира избран от нас темплейт, извършвайки многомерна минимизация. Избираме да работим с елипсовиден темплейт с Гсаусова дебелина, описван от следните параметри: централна позиция $(x_0,y_0)$, диаметър $d$ и елиптичност $\tau$, свързани с двете пулооси посредством $d = 2\sqrt{ab}$, $\tau = 1 - b/a$, дебелина $\mathcal{\omega} = 2\sqrt{2\ln2}\sigma$ и ъгъла на ориентация на голямата пулоос $\xi_\tau$. За да отчетем асиметрията на реконструкциите, предизвикана от ефекта на Доплер, към темплейта добавяме и излъчваща арка, дефинирана като:
\begin{equation}
	S(x,y;s,\xi_s) = N_0(1 + s\cos(\phi - \xi_s)),
\end{equation}
където $s$ е относителният интензитет на арката и $\xi_s$ е ъгълът ѝ на ориентацията. Параметърът $N_0$ служи за нормиране на (8.10) към единица. С това можем да запишем израз за пълният темплейт:
\begin{equation}
	h(x,y) = S(x,y;s,\xi_s)e^{-\frac{(d(x,y))^2}{2\sigma^2}}\in[0,1]
\end{equation}
където $d(x,y)$ е минималното разстояние между точката с координати $(x,y)$, и елипсата с параметри $\{d_0,\tau,x_0,y_0\}$. Темплейтът, най добре описващ дадена реконструкция приемаме за този, който минимизира \emph{дивергенцията на Батачаря} \footnote{Тази величина е първоначално въведена като мярка за приликата между две вероятностни разпределения - $Bh(a(x),b(x)) = 0$ означава, че $a(x) = b(x)$, докато $Bh(a(x), b(x)) = \infty$, че $a(x)$ и $b(x)$ са "ортогонални"$\,$ в смисъл, че сечението на двете разпределения е нулево.}:
\begin{equation}
	Bh\left(I(x,y)||h(x,y)\right)= -\log\int\sqrt{I(x,y)h(x,y)}dxdy
\end{equation}