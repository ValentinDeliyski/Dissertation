\section{Увод}

Тематиката на тази дисертация стъпва върху резултатите на колаборацията Event Horizon Telescope (EHT), която за пръв път успява да постигне наблюдателна резолюция, достатъчна за \emph{директното} заснемане на непосредствената околност на свръхкопактните обекти в ядрата на галактиката M87 и Млечният път \cite{EHT_M87_I} - \cite{EHT_SGR_VIII}. Тези и бъдещи такива наблюдателни постижения ще играят централна роля в изследванията на природата на гравитацията в режим на най-силните полета. Те могат да допринесат за експерименталното потвърждаване на присъствието на нови фундаментални полета, както и за съществуването на \emph{екзотични компактни обекти}, като пространствено-времеви тунели или голи сингулярности. Подобни на тях обекти произлизат естествено от обобщени теории на гравитацията, което прави наблюдателното им засичане от фундаментално значение.\\

\emph{Целта на този дисертационен труд е да изследва възможността за различаването на подобни екзотични компактни обекти от черни дупки в Общата Теория на Относителността (ОТО), чрез съществуващите и бъдещи наблюдения на колаборацията EHT.}\\

\noindent По-конкретно, този дисертационен труд се фокусира върху отличителните наблюдателни особености на тези обекти, които в най-общ смисъл може да се проявят чрез три наблюдаеми характеристики: морфологията на получените образи, променливостта им, и поляризацията на лъчението им. Те носят със себе си информация за природата на компактният обект, както и за гравитационната теория която го описа, с усложнението, че тя е нелинейно зацепена към магнито-хидродинамаичните процеси на излъчващата среда. Това прави интерпретацията на съществуващи и бъдещи наблюдения силно нетривиална задача, и е едно от основните затруднения в изследователския проблем.\\

\noindent Друго затруднение е, че много широк клас от съществено различни екзотични компактни обекти, могат да оставят качествено сходен отпечатък в наблюденията. Например въпреки, че наблюдаваната сянка на M87$^*$ е съвместима с тази на черна дупка на Шварцшилд \cite{EHT_M87_I}, имайки предвид независимите оценки за масата на обекта \cite{Gebhardt_2011}, съществува широк клас от екзотични компактни обекти, чиято сянка е достатъчно морфологично сходна (а за някой дори \emph{стриктно} идентична \cite{PhysRevD.103.084040}), за да бъде съвместима с тези наблюдения.\\

\noindent От друга страна наблюдателният отпечатък на обекти, притежаващи качествено \emph{различна} оптична проява, вследствие на ефекта на гравитационната леща, може да бъде подтиснат от крайната разделителна способност, или други \emph{технически} предизвикателства на наблюденията. Това прави поставената цел комплексна задача, свързваща теорията с експеримента. Подходът ни е следният:\\

\textbf{1)} Започваме с изследването на оптичните прояви на избрани "представители"$\,$ на екзотични компактни обекти, чиято природа се различава силно от тази на черните дупки в ОТО - пространствено-времеви тунели и голи сингулярности. В \cite{Delijski2022}, и разширено в глава 6, изследваме морфологията на образите на излъчващата среда, генерирани от тези обекти. Показваме, че при определени стойности на техните метрични параметри, оптичната им проява е \emph{съществено различна} от тази на черни дупки в ОТО. Намираме, че те притежават набор от концентрични пръстеновидни образи (които ще наричаме \emph{екзотични}), разположени където би била сянката на черните дупки в ОТО. Тези пръстени биха служили като \emph{ясен и еднозначен белег} за отклонение на гравитационната теория от ОТО, ако бъдат наблюдавани. С това възниква естественият въпрос - до каква степен тези образи са наблюдаеми? Използваме идеализиран модел на геометрично тънък и оптически плътен акреционен диск \cite{Page1973} за да покажем, че наблюдаваният поток от тези образи е съразмерим с максимума на цялото изображение, и че това се запазва при по-реалистични модели на излъчващата среда в \cite{Deliyski2024} и глава 8.\\

\textbf{2)} Паралелно с това, мотивирани от резултатите \cite{EHT_M87_VII}, \cite{EHT_M87_VIII} и \cite{EHT_SGR_VII}, \cite{EHT_SGR_VIII}, изследваме до каква степен природата на пространство-времето се отпечатва върху поляризацията на получените образи \cite{Delijski2022}, \cite{Deliyski2023}. Стъпваме на предишни разглеждания, базирайки се на прост аналитичен модел на излъчването \cite{Narayan2021} \cite{Gelles2021}, за да покажем, че директните поляризирани образи на излъчващата среда се влияят слабо от природата на централният обект, и гравитационната теория която го описва. Показваме също обаче, че релативистките индиректни образи се влияят \emph{силно} от това. Относителните отклонения в интензитета на поляризираното лъчение, спрямо черни дупки на Шварцшилд, могат да достигнат порядък. С това показваме, че ако наблюденията са способни на разделят директните от индиректните образи, тяхната поляризация може да служи като допълнително ограничение върху природата на централният компактен обект.\\

\textbf{3)} Мотивирани от резултатите, представени в \cite{Deliyski2022}, \cite{Delijski2022} и \cite{Deliyski2023} (съответно Глави 6, 7 и 8 на дисертацията), както и от подобни изследвания \cite{Eichhorn2022}, \cite{Qin2021}, \cite{Geometric_Modeling}, разглеждаме способността на съвременните и бъдещи наблюдения на колаборацията EHT да засекат предсказаните тук ефекти. Показваме, че морфологията на екзотичните образи се губи при реални наблюдения (следствие на крайната ефективна разделителна способност), но те все пак оставят отпечатък, под формата на повишено фоново излъчване в централната депресия на крайните наблюдения, спрямо черни дупки в ОТО. Намираме, че при разширение на набора от телескопи, използвани за наблюденията, тази количествена разлика нараства до два порядъка. Също така показваме, че повишаване на наблюдателната честота от $230$ GHz (използвана за всички досегашни резултати на EHT) до 345 GHz (планираната за бъдещи наблюдения), прави наблюденията значително по-чувствителни към наличието на екзотични образи. При тази честота намираме, че наблюдаваните образи придобиват локалани максимуми в централната депресия, чиито интензитет може да достигне 30\% от максимума за цялото изображение. Това би служило като \emph{ясен и еднозначен} белег, за отклонение на гравитационната теория, описващата наблюдавания обект, от ОТО.

\subsection{Структура на дисертационният труд}