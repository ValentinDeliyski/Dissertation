\section{Увод}

Тематиката на тази дисертация стъпва върху резултатите на колаборацията Event Horizon Telescope (EHT), която за пръв път успява да постигне наблюдателна резолюция, достатъчна за \emph{директното} заснемане на непосредствената околност на свръхкомпактните обекти в ядрата на галактиката M87 и Млечният път \cite{EHT_M87_I} - \cite{EHT_SGR_VIII}. Тези и бъдещи такива наблюдателни постижения ще играят централна роля в изследванията на природата на гравитацията в режим на най-силните полета. Те могат да допринесат за експерименталното потвърждаване на присъствието на нови фундаментални полета, както и за съществуването на \emph{екзотични компактни обекти}, като пространствено-времеви тунели или голи сингулярности. Подобни на тях обекти произлизат естествено от обобщени теории на гравитацията, което прави наблюдателното им засичане от фундаментално значение.\\

\emph{Целта на този дисертационен труд е да изследва възможността за различаването на подобни екзотични компактни обекти от черни дупки, чрез съвременните и бъдещи наблюдения на колаборацията EHT.}\\

\noindent По-конкретно, ние се фокусираме върху отличителните наблюдателни особености на тези обекти, които в най-общ смисъл може да се проявят чрез следните три наблюдаеми характеристики: морфологията на получените образи, променливостта им, и поляризацията на лъчението им. Те носят със себе си информация за природата на компактният обект, както и за гравитационната теория която го описва, с усложнението, че тя е нелинейно зацепена към магнито-хидродинамичните процеси на излъчващата среда. Това прави интерпретацията на съществуващи и бъдещи наблюдения силно нетривиална задача, и е едно от основните затруднения в изследователския проблем.\\

\noindent Друго затруднение е, че много широк клас от съществено различни екзотични компактни обекти, могат да оставят качествено сходен отпечатък в наблюденията. Например въпреки, че наблюдаваната сянка на M87$^*$ е съвместима с тази на черна дупка на Шварцшилд \cite{EHT_M87_I}, имайки предвид независимите оценки за масата на обекта \cite{Gebhardt_2011}, съществува широк клас от екзотични компактни обекти, чиято сянка е достатъчно морфологично сходна (а за някои дори \emph{строго} идентична \cite{PhysRevD.103.084040}), за да бъде съвместима с тези наблюдения.\\

\noindent От друга страна наблюдателният отпечатък на обекти, притежаващи качествено \emph{различна} оптична проява, вследствие на ефекта на гравитационната леща, може да бъде подтиснат от крайната разделителна способност, или други \emph{технически} предизвикателства на наблюденията. Това прави поставената цел комплексна задача, свързваща теорията с експеримента.\\

\noindent За да дадем допълнителен фокус на целта на дисертацията, приемаме следната работна хипотеза:\\

\emph{Наблюденията на колаборацията EHT, през 2017 г., могат да бъдат възпроизведени от синхотронно излъчваща плазма, около свръхмасивни компактни обекти които \textbf{не} притежават хоризонт на събитията.}\\

\noindent Имайки това в предвид, ние все пак очакваме природата на компактните обекти да се "отпечата"$\,$ върху горе-споменатите три наблюдаеми характеристики на образите. Работата по постигане на целта на дисертацията е публикувана под формата на четири оригинални публикации (белязани I, II, III и IV в текста). Методиката ни има следната логическа последователност:\\

\textbf{1)} Започваме с изследването на оптичните прояви на избрани "представители"$\,$ на екзотични компактни обекти, чиято природа се различава силно от тази на черните дупки - пространствено-времеви тунели и голи сингулярности. В публикация I, и разширено в глава 5 от дисертацията, изследваме морфологията на образите на излъчващата среда, генерирани от подобни обекти. Показваме, че при определени стойности на техните метрични параметри, оптичната им проява е \emph{съществено различна} от тази на черни дупки. Намираме, че те притежават набор от концентрични пръстеновидни образи (които ще наричаме \emph{екзотични}), разположени където би била сянката на черните дупки. Тези пръстени биха служили като ясен и еднозначен белег за съществуването на подобен тип обекти, ако бъдат наблюдавани. С това възниква естественият въпрос - до каква степен тези образи са наблюдаеми? Използваме идеализиран модел на геометрично тънък и оптически плътен акреционен диск \cite{Page1973}, за да покажем, че наблюдаваният поток от тези образи е съразмерим с максимума на цялото изображение, и че това се запазва при по-реалистични модели на излъчващата среда в публикация IV и глава \emph{7} от дисертацията.\\

\textbf{2)} Паралелно с това, мотивирани от резултатите \cite{EHT_M87_VII}, \cite{EHT_M87_VIII} и \cite{EHT_SGR_VII}, \cite{EHT_SGR_VIII}, в публикации II и III изследваме до каква степен природата на пространство-времето се отпечатва върху поляризацията на получените образи. Стъпваме на предишни разглеждания, базирайки се на прост аналитичен модел на излъчването \cite{Narayan2021} \cite{Gelles2021}, за да покажем, че директните поляризирани образи на излъчващата среда се влияят слабо от природата на централният обект, и гравитационната теория която го описва. Показваме също обаче, че релативистките индиректни образи се влияят \emph{силно} от това. Относителните отклонения в интензитета на поляризираното лъчение, спрямо черни дупки на Шварцшилд, могат да достигнат до един порядък. С това показваме, че ако наблюденията са способни на разделят директните от индиректните образи, тяхната поляризация може да служи като допълнително ограничение върху природата на централният компактен обект.\\

\textbf{3)} Мотивирани от резултатите, представени в публикации I, II и III (съответно глави \emph{5} и \emph{6}), както и от подобни изследвания \cite{Eichhorn2022}, \cite{Qin2021}, \cite{Geometric_Modeling}, разглеждаме способността на съвременните и бъдещи наблюдения на колаборацията EHT да засекат предсказаните тук ефекти в публикация IV. Показваме, че морфологията на екзотичните образи се губи при настоящите наблюдателни условия (следствие на крайната ефективна разделителна способност), но те все пак оставят отпечатък, под формата на повишено фоново излъчване в централната депресия на крайните наблюдения, спрямо черни дупки. Намираме, че при разширение на набора от телескопи, използвани за наблюденията, тази количествена разлика нараства до два порядъка. Също така показваме, че повишаване на наблюдателната честота от $230$ GHz (използвана за всички досегашни резултати на EHT) до 345 GHz (планираната за бъдещи наблюдения), прави наблюденията значително по-чувствителни към наличието на екзотични образи. При тази честота намираме, че наблюдаваните образи придобиват локални максимуми в централната депресия, чиито интензитет може да достигне 30\% от максимума за цялото изображение. Това е наблюдателно предсказание което може да бъде вземано в предвид за бъдещи наблюдения.

\subsection{Структура на дисертационният труд}

Самата дисертация може да се разглежда като разделена да две части - обща и специализирана.\\

Общата част има за цел да предостави на читателя основният контекст и физическа основа на разглежданата тематика. Тя обхваща глави \emph{2} до \emph{4} и има следната структура: Глава \emph{2} представя основните закони за разпространението на електромагнитното лъчение в изкривено пространство-време. Извеждаме т.н. приближение на геометричната оптика, в рамките на което се разглеждат всички оптически ефекти в гравитацията. Представяме също и общия вид на динамичните уравнения на светлинните лъчи (в рамките на геометричната оптика), както и ковариантното уравнение за поляризиран лъчист пренос. В глава \emph{3} представяме основните наблюдателни резултати на колаборацията EHT, върху които базираме нашите изследвания. В глава \emph{4} представяме разглежданите от нас екзотични компактни обекти, както и техните основни свойства.\\

\noindent Техническата част покрива глави \emph{5} до \emph{7} и представлява изложение на оригиналните резултати на автора. Глава \emph{5} представя в разширен вид изследванията от публикация I, както и обзор на еквивалентни изследвания за голи сингулярности \cite{Gyulchev2020}, \cite{Gyulchev2021}. Глава \emph{6} представя резултатите от публикации II и III. Глава \emph{7} представя резултатите от публикация IV, и последно глава \emph{8} представлява заключение и обзор на основният научен принос на автора.\\

\noindent Освен тези глави, дисертационният труд има четири допълнения, обхващащи някой аспекти на поставената научна цел, които заслужават допълнително внимание: Допълнение \emph{A} представя извод, от първи принципи, на функцията на излъчване на топлинно разпределена релативистка плазма. Подобно пълно извеждане почти не се намира в литературата. Допълнение \emph{Б} представя разработения от автора числен код, използван в публикации II, III и IV. Допълнение \emph{В} представя използваният из дисертацията локален ортонормиран базис, и последно допълнение \emph{Г} представя извод на уравненията, определящи позицията на последните стабилни и свързани кръгови орбити за масивни частици.