\section{Обзор на научният принос}

Научният принос на тази дисертация стъпва върху представените в глава 3 резултати и най-общо казано има за цел да оцени колко информация за фоновото пространство-време може да бъде извлечена от подобни наблюдения.\\

 \emph{Допускаме хипотезата, че наблюденията, представени в глава 3, могат да бъдат възпроизведени от синхотронно излъчваща плазма, около свръхмасивни компактни обекти които \textbf{не} притежават хоризонт на събитията}.\\\newline
Произведените оригинални научни публикации, които изследват последствията от тази хипотеза CITE OWN PAPERS HERE, имат следната логическа последователност: \\

\noindent$\bullet$ В CITE PROCEEDING HERE изследваме морфологията на релативистките образи на тънки, екваториални и симетрични\footnote{ В случаят "симетричен" диск означава, че пространствено-времевият тунел притежава акрекционен диск от \emph{двете} страни на гърловината си.} акреционни дискове около статични и стационарни пространствено-времеви тунели. Въпреки факта, че реалната излъчваща среда се очаква да се разпростира значително извън екваториалната равнина, основните морфологични характеристики на образите ѝ се запазват и при разглеждане на чисто екваториално излъчване. Показваме, че разгледаният клас пространствено времеви тунели генерират характерни "двойни"$\,$ образи за всеки порядък. Те са разположени \emph{във вътрешността} на сянката\footnote{Тук терминът "сянка"$\,$ трябва да се разбира като компактна област в параметричното пространство на прицелни параметри на фотоните такива, че те падат върху гърловината на тунела. За разлика от черните дупки обаче, тези фотони преминават през тази гърловина и се разсейват от другата ѝ страна. За далечен наблюдател това изглежда като лъчение идващо от вътрешността на сянката.} и съответстват на фотони, излъчени от отсрещната страна на гърловината и преминали през нея, за да достигнат наблюдателя. Подобна структура в образите изцяло липсва при класическите черни дупки на Кер-Нюман и следователно могат да служат са силни наблюдателни белези за съществуването на по-екзотични компактни обекти. Ще съпоставим този резултат с подобни изследвания за голи сингулярности, на базата на което ще приемем следното:\\

\emph{Голям набор от екзотични компактни обекти, които \textbf{не} притежават хоризонт на събитията могат да генерират релативистки образи на излъчващата си среда, чиято морфология силно се различава от тази на черните дупки.}\\

\noindent Наличието на тези образи повдига два въпроса:\\\newline

\noindent\textbf{1)} Съвременните наблюдателни техники имат ли разделителната способност да засекат присъствието на тези допълнителни образи?\newline

\noindent\textbf{2)} Имайки предвид, че един от основните източници на информация за физиката на излъчващата среда е поляризацията на полученото лъчение, може ли природата на централният компактен обект да се "кодира"$\,$ в тази поляризация?\\\newline
$\bullet$ В CITE THE TWO POL PAPERS HERE сме разгледали опростен аналитичен модел на поляризацията, излъчена от тънък екваториален диск (следвайки CITE COLAB HERE). Показваме, че и за двата типа компактни обекти (тунели в CITE WORMHOLE PAPER и голи сингуларности CITE SINGULARITY PAPER):\\\newline
\textbf{1)} Поляризацията на класическите\footnote{Тук под "класически"$\,$ имаме предвид такива образи, чиито еквивалент съществува за черни дупки на Кер. Такива биха били директният образ и релативистките пръстени, извън сянката. За разлика от тях "екзотичните"$\,$ образи биха били тези, генерирани вътре в сянката на обекта.} директни образи се влияят слабо от природата на централният обект - геометрията на магнитното поле доминира наблюдаваната структура на поляризацията.\\\newline
\textbf{2)} Поляризацията на класическите релативистки образи се влияе силно от природата на централният обект. В зависимост от геометрията на магнитното поле, относителните отклоненията на поляризационната структура от тази за черни дупки на Кер може да достигнат 
\emph{порядък.}\\\newline
\textbf{3)} Интензитетът на екзотичните образи, генерирани от компактните обекти могат да бъдат значително по-високи от тези на класическите образи от същият порядък.\\\newline
\textbf{Изводите които правим от тези изследвания са следните:}\\
\emph{Класическите директни образи могат да се използват за фиксиране на геометрията на магнитното поле в излъчващата среда (понеже се влияят слабо от природата на централният обект). Това поле тогава може да се използва за възпроизвеждане на поляризацията на релативистките образи, предполагайки дадено пространство-време. Ако наблюденията покажат силни отклонения между измерената структура на поляризацията, и пресметнатата, базирана на магнитно поле извлечено от директните образи, тогава можем да съдим, че централният обект не се описва с първоначално приетата метрика.}\\\newline
Този подход има голямото предимство да различава между големи класове пространство-времена, на базата само на информацията от равновесно лъчение! Не изисква дългосрочни наблюдения на орбиталното движение на други обекти, или наличието на едромащабни структури като джетове (каквито липсват за Sgr A$^*$).\\

\noindent Големият му недостатък обаче, е нуждата от достатъчно висока разделителна способност за различаване на директни от релативистки образи. \\\newline
$\bullet$ Това мотивира разглежданията ни в CITE SIM PAPER HERE. Изследваме способността за разделянето на екзотичните образи, генерирани от голи сингуларности. Показваме, че наблюденията на EHT през 2017, заедно с използваният от тях модел за реконструкция, \textbf{не} са способни да разделят тези образи. За сметка на това те биха засекли по-висок интензитет в централната депресия на реконструираният образ на компактният обект. Използваме известният вече в литературата методи за темплейтен анализ CITE TEMPLATING PAPERS HERE, с помощта на който характеризираме количествено "дълбочината"$\,$ на централната депресия. Въвеждаме използваният в литературата параметър $f_c$, представляващ отношението на минималният поток в централната депресия, и средният такъв за пръстеновидната структура. Показваме, че този параметър, оценен за голи сингуларности, може да се различава с порядък от черни дупки на Кер. Тази разлика се увеличава с разширяването на броя радио телескопи, участващи в наблюдението. Показваме също, че въвеждането на втора, по-висока, наблюдателна честота позволява засичането на локални максимуми в централната депресия. Подобна морфология изцяло липсва при черни дупки на Кер, и следователно би служила като \emph{еднозначен} белег за наличието на допълнителни екзотични образи.\\\newline
В следващите глави подробно ще представим обобщеният тук оригиналният труд, както и допълнителни резултати, на базата на които се подготвят следващи публикации. 
\lfoot{}