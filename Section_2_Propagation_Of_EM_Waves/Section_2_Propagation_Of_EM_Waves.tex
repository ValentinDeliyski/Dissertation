\section{Разпространение на електромагнитни вълни в изкривено пространство време}

Всички разглеждания в настоящият труд се крепят върху теорията за разпространение на електромагнитното лъчение в изкривено пространство време. Затова ще отделим специално внимание върху стандартното приближение, в рамките на което се разглеждат всички оптични ефекти в изкривено пространство време. Това е т.н. \emph{приближение на геометричната оптика} \cite{Schneider1992}. \\

Тази глава е разделена на три части. В първата ще представим уравненията на Максуел в ковариантна форма върху фоново пространство-време $\mathcal{M}$ с метрика $g$. Ще покажем как в приближение на високи честоти и ниски амплитуди на вълната (WKB приближение) обратната реакция на лъчението върху пространство-времето може да бъде пренебрегната. Това позволява електромагнитното лъчение да бъде третирано като разпространяващо се по \emph{изотропни геодезични линии} и пренасящо паралелно вектора си на поляризацията по тези линии \cite{Dolan2018}, \cite{Oancea2020}.\\

Във втората ще покажем общата форма на динамичните уравнения, описващи разпростирането на електромагнитните лъчи.\\

И последно в третата част ще представим в ковариантна форма уравнението за лъчист пренос през излъчваща и поглъщаща среда, което играе централна роля в представените в глава \emph{8} резултати.

\subsection{Решаване на уравненията на Максуел в термини на WKB приближение}

Уравненията на Максуел в ковариантна форма имат следният вид \cite{Weinberg1972}:

\begin{subequations}
	\begin{align}
	&\nabla_\mu F^{\mu\nu} = \mu_0 j^\nu,\quad \nabla_{[\mu}\tilde{F}_{\mu\nu]} = 0,\\
	&F^{\mu\nu} = 2\nabla^{[\mu}A^{\nu]}, \quad \tilde{F}_{\mu\nu} = \frac{1}{2}\epsilon_{\mu\nu\alpha\beta}F^{\alpha\beta},
	\end{align}
\end{subequations}
където $A_\mu$ e калибровачният потенциал на електромагнитното поле и $\epsilon_{\mu\nu\alpha\beta}$ e четири-тензорът на Леви-Чевита. Записваме нехомогенното уравнение явно чрез $A^\mu$:

\begin{equation}
	\nabla_\mu \nabla^\mu A^\nu - \nabla_\mu \nabla^\nu A^\mu = \mu_0 J^\nu.
\end{equation}
След което използваме дефинициите на тензорите на Риман и Ричи:
\begin{equation}
	\nabla_\mu \nabla_\nu A^\alpha - \nabla_\nu \nabla_\mu A^\alpha = R^\alpha_{\,\,\,\mu\nu\beta} A^\beta, \quad R_{\mu\beta} = R^\beta_{\,\,\,\mu\beta\gamma},
\end{equation} 
заедно с калибровката на Лоренц $\nabla_{\mu} A^\mu = 0$ можем да запишем вълново уравнение за $A^\mu$:
\begin{equation}\label{wave_A^mu}
	\nabla_{\mu}\nabla^\mu A^\nu + R^\nu_{\,\,\,\gamma} A^\gamma = \mu_0 J^\nu
\end{equation}
Членът $R^\nu_{\,\,\,\gamma} A^\gamma$ в уравнение (\ref{wave_A^mu}) показва явно обратната връзка, която електромагнитното поле оказва върху пространство времето. Присъствието му прави зацепването на уравненията на Максуел към полевите уравнения на Айнщайн-Хилбърт сложно и намирането на общи решения възможно само в рамките на приближения. Стандартна практика при разглеждането на оптични ефекти в изкривено пространство-време и да се направи следният анзац за калибровачния потенциал $A^\mu$ \cite{Oancea2020}\cite{Misner1973}:

\begin{equation}\label{A^mu_anzatz}
	A^\mu = \text{Re}\left\{a^\mu(x^\mu)e^{i\frac{\Psi(x^\mu)}{\epsilon}}\right\},
\end{equation}
където $\epsilon$ e малък параметър. Същността на WKB приближението идва в представянето на комплексната амплитуда $a^\mu(x^\mu)$ и реалната фаза $\Psi(x^\mu)$ като степенен ред по $\epsilon$:

\begin{subequations}
	\begin{align}
		&a^\mu(x^\mu) = a_0^\mu(x^\mu) + \epsilon a_1^\mu(x^\mu) + \mathcal{O}(\epsilon^2),\\
		&\Psi(x^\mu) = \Psi_0(x^\mu) + \epsilon \Psi_1(x^\mu) + \mathcal{O}(\epsilon^2).
	\end{align}
\end{subequations}
Тук компонентата $\Psi_0(x^\mu)$ се интерпретира като бързо менящата се компонента на фазата, докато компонентите $\Psi_i(x^\mu)$, пропорционални на степените на $\epsilon$, се приемат за бавно менящи се. Аналогично компонентата $a_0^\mu(x^\mu)$ се интерпретира като достатъчно малка по абсолютна стойност за да не предизвика силна обратна реакция върху фоновото пространство и бавно меняща се с времето. Компонентите $a_i^\mu(x^\mu)$, пропорционални на $\epsilon$, се приемат за бързо-менящи се. Замествайки (\ref{A^mu_anzatz}) в уравнение (\ref{wave_A^mu}) получаваме:
\begin{equation}
	\begin{split}
	&\text{Re}\bigg\{ e^{i\frac{\Psi}{\epsilon}}  \bigg[  \nabla_{\mu}\nabla^\mu\{a_0^\nu + \epsilon a_1^\nu + \mathcal{O}(\epsilon^2) \} + \\
	& + \frac{2i}{\epsilon} \nabla^{\mu}\left( a_0^\nu + \epsilon a_1^\nu + \mathcal{O}(\epsilon^2) \right) \nabla_\mu \left\{\Psi_0 + \epsilon \Psi_1 + \mathcal{O}(\epsilon^2) \right\} + \\
	 & + \frac{i}{\epsilon} \left(a_0^\nu + \epsilon a_1^\nu + \mathcal{O}(\epsilon^2) \right) \nabla_\mu\nabla^\mu\left\{\Psi_0 + \epsilon \Psi_1 + \mathcal{O}(\epsilon^2) \right\} - \\
	 & - \frac{1}{\epsilon^2} \left(a_0^\nu + \epsilon a_1^\nu + \mathcal{O}(\epsilon^2) \right) \nabla^\mu\left\{ \Psi_0 + \epsilon \Psi_1 + \mathcal{O}(\epsilon^2) \right\}\nabla_\mu\left\{\Psi_0 + \epsilon \Psi_1 + \mathcal{O}(\epsilon^2)\right\} + \\
	 & + R^\nu_{\,\,\,\gamma} \left( a_0^\gamma + \epsilon a_1^\gamma + \mathcal{O}(\epsilon^2) \right) \bigg] \bigg\}  = \mu_0 J^\nu
	\end{split}
\end{equation}
Също така от калибровката на Лоренц $\nabla_\mu A^\mu = 0$ имаме:
\begin{equation}
	\text{Re}\big\{\nabla_\mu\left(a_0^\nu + \epsilon a_1^\nu + \mathcal{O}(\epsilon^2)\right) + \frac{i}{\epsilon}\left(a_0^\nu + \epsilon a_1^\nu + \mathcal{O}(\epsilon^2)\right)\nabla_\mu\left(\Psi_0 + \epsilon \Psi_1 + \mathcal{O}(\epsilon^2) \right)\big\} = 0
\end{equation}

\subsubsection{Условие за изотропност}

Решаваме първо уравнения (7) и (8) с точност до $\epsilon^{-2}$ и намираме:

\begin{equation}
	\text{Re}\big\{a^\nu_0\nabla_\mu\Psi_0\nabla^\mu\Psi_0\big\} = 0\quad \xrightarrow{a^\nu_0\ne 0}\quad k_{0,\mu} k^\mu_0 = 0,
\end{equation}
което представлява условието за изотропност с въвеждането на вълновият вектор $k_\mu := \nabla_\mu\Psi$. Физическата интерпретация на този резултат е, че в границата на безкрайно високи честоти, електромагнитните вълни се разпространяват по изотропни геодезични линии.

\subsubsection{Условие за паралелен пренос на поляризационният вектор}

Сега решаваме уравнения (7) и (8) с точност до $\epsilon^{-1}$. Имаме:

\begin{equation}
	\text{Re}\big\{2 k_0^\mu\nabla_\mu a_0^\nu + a_0^\mu\nabla_\mu k^\mu_0 - a_1^\nu k_{0,\mu} k^\mu_0 - 2 a_0^\nu k_{0,\mu}k^\mu_1\big\} = 0.
\end{equation}
Третият член се нулира от условието за изотропност. Извършваме контракция на уравнение (10) с $a^*_{0,\nu}$ и означавайки $|a_0|^2 := a^\nu_0 a^*_{0,\nu}$ получаваме:

\begin{equation}
	\nabla_\mu\left\{k_0^\mu |a_0|^2\right\} = 2  |a_0|^2 k_{0,\mu}k_1^\mu,
\end{equation}
което представлява транспортното уравнение за потока. Нехомогенният член $2  |a_0|^2 k_{0,\mu}k_1^\mu$ отчита появата на дисперсионен ефект.
От калибровката на Лоренц намираме с точност $\epsilon^{-1}$:
\begin{equation}\label{a_orthog_k}
	a_0^\mu k_{0,\mu} = 0.
\end{equation}
Използвайки това дефинираме вектор на поляризацията $a^\mu = |a|f^\mu$, който е ортогонален на вълновия вектор. Тогава ако запишем уравнение (2.10) в термини на $f^\mu$ получаваме:

\begin{equation}
	2|a_0|k_0^\mu\nabla_\mu f^\nu + \frac{f^\nu_0}{|a_0|}\left[\nabla_\mu\left\{k_0^\mu |a_0|^2\right\} - 2  |a_0|^2 k_{0,\mu}k_1^\mu\right] = 0,
\end{equation}
от където следва условието за паралелен пренос на вектора на поляризацията с точност $\epsilon^{-1}$:
\begin{equation}
	k^\mu\nabla_\mu f^\nu = 0.
\end{equation}
Важно е да се отбележи, че въпреки $f^\mu$ да е четири-компонентен вектор, той притежава само две (комплексни) степени на свобода. Това се налага от калибровката на Лоренц $\nabla_\mu A^\mu = 0$ и самата дефиниция на $f^\mu$, от която следва, че $f^\mu f_\mu = 1$.

\subsubsection{Коментар върху областта на приложимост на WKB приближението}

Нека разгледаме по-подробно наложените приближения върху фазата и амплитудата на калибровачния потенциал:\\

1) \textbf{Приближението за малка амплитуда.} Това идва от наличието на члена на взаимодействие $R^\nu_{\,\,\,\gamma} A^\gamma$. Количеството енергия, което се пренася от една електромагнитна вълна е, във всички практически случаи, абсолютно пренебрежимо спрямо характерните енергии, нужни за въздействие върху пространство-времето. Следва, че това приближение е валидно за електромагнитни вълни във \emph{всички} режими.\\

2) \textbf{Приближение са бързо меняща се фаза}. То може да бъде изказано по по-физически ясен начин:\\

\emph{Дължината на вълната на електромагнитните вълни е пренебрежимо малка спрямо характерният радиус на кривината на пространство-времето.}\\

Тъй като ни интересуват оптични ефекти в околност на свръхмасивни компактни обекти, за характерен радиус можем да вземем радиуса на Шварцшилд на дадения обект. Характерните дължини на вълните на наблюденията които ни интересуват са в радио диапазона\footnote{Честотите, на които се правят съвременните наблюдения на EHT колаборацията, са между 86 GHz и 345 GHz. Това съответства на дължини на вълните $\approx 10^{-3}$ m}. Сравнявайки ги с радиуса на Шварцшилд на свръхмасивен компактен обект\footnote{За маси от порядък $10^6 M_\odot$, това съответства на $R_{\text{Schw}}\approx 10^9$ m}, можем да се убедим, че и това приближение е изпълнено с огромна точност. Именно тази точност ни позволява да пренебрегнем и нехомогенният член в уравнение (11) и да го приемем като закон за запазване на потока:

\begin{equation}
		\nabla_\mu\left\{k_0^\mu |a_0|^2\right\} = 0
\end{equation}

\subsection{Динамични уравнения за разпространение на светлинните лъчи}
За да коментираме върху динамиката на електромагнитните вълни в приближението на геометричната оптика, нека първо конкретизираме какъв тип пространство-време ще разглеждаме в тази дисертация. Избираме да конкретизираме разглежданията си върху стационарни и аксиално симетрични пространства (те бивайки най-релевантните от астрофизическа гледна точка). Техният линеен елемент в стандартните псевдо-сферични координати $\{t,r,\theta,\phi\}$ може да се запише като:
\begin{equation}
	ds^2 = g_{tt}dt^2 + g_{rr}dr^2 + 2g_{t\phi}dtd\phi + g_{\theta\theta}d\theta^2 + g_{\phi\phi} d\phi^2. 
\end{equation}
Тези пространства притежават два Килингови вектора $\xi = \partial_t$ и $\eta = \partial_\phi$, които според теоремата на Ньотер\footnote{Това може да се види директно от уравненията на Хамилтон:
	\begin{equation*}
		\begin{aligned}
			&\frac{dk_t}{d\lambda} = - \frac{\partial H}{\partial t} = 0 \rightarrow k_t \equiv -E = \text{const}\\
			&\frac{dk_\phi}{d\lambda} = - \frac{\partial H}{\partial \phi} = 0 \rightarrow k_\phi \equiv L_z = \text{const}\\
		\end{aligned}
\end{equation*}}, съответстват на запазващи се величини:
\begin{subequations}
	\begin{align}
		&k_t = \text{const} \equiv -E\\
		&k_\phi = \text{const} \equiv L_z.
	\end{align}
\end{subequations}
Освен тази явна симетрия обаче, разглежданите от нас метрики (виж глава \emph{5}) също притежават и т.н. \emph{скрита симетрия}, под формата на тензор на Килинг $K_{\mu\nu}$. Наличието на тази допълнителна симетрия значително улеснява аналитичните разглеждания. По-конкретно, тя позволява разделянето на променливите в динамичното уравнение на Хамилтон-Якоби:
\begin{equation}
	\frac{\partial S}{d\lambda} + H\left(x^\mu,\frac{\partial S}{\partial x^\mu}\right) = 0, \,k_\mu = \frac{\partial S}{\partial x^\mu},\, H = \frac{1}{2}g^{\mu\nu}k_\mu k_\nu.
\end{equation}
Анзацът за на решението, възползващо се от Килинговата симетрия е:
\begin{equation}
	S(t,r,\theta,\phi) = \frac{1}{2}m^2\lambda -Et + L_z\phi + S_r(r) + S_\theta(\theta),
\end{equation}

\noindent където за пълнота още не сме положили масата на частиците $m = 0$. Този анзац води до следната обща форма на динамичните уравнения на лъчите \cite{Chandrasekhar}\cite{Wang2019}\cite{Gyulchev2018}:
\begin{subequations}
	\begin{align}
		&\frac{dt}{d\lambda} = g^{t\mu}k_\mu = -g^{tt}E + g^{t\phi}L_z\\
		&\frac{dr}{d\lambda} = \pm g^{rr}\sqrt{R(r)}\\
		&\frac{d\theta}{d\lambda} = \pm g^{\theta\theta}\sqrt{\Theta(\theta)}\\
		&\frac{d\phi}{d\lambda} = g^{\phi\mu}k_\mu = g^{\phi\phi}L_z - g^{\phi t}E.
	\end{align}
\end{subequations}
Ще базираме аналитичните си разглеждания в глава \emph{5} на тези уравнения.\\

От основен интерес за нас ще бъде факта, че уравнения (2.20), под определени условия, допускат съществуването на \emph{кръгови} фотонни орбити \cite{Teo}. Те могат да се разглеждат като граничният случай, разделящ множествата от фотони които, налитайки към централният обект, се разсейват, и тези които попадат върху него. Следователно свойствата на тези орбити определят и видимата форма на обекта за далечен наблюдател - т.н. \emph{сянка}.\\

От числена гледна точка обаче, системата (2.20) е неудобна. Наличието на знакът $\pm$ пред (2.20б) и (2.20в), както и присъствието на корен от функции, притежаващи нули в някой характерни точки, прави численото им интегриране сложно от имплементационна гледна точка. Много по-удобно представяне на динамиката в този контекст, са уравненията на Хамилтон с $H = \frac{1}{2}g^{\mu\nu}k_\mu k_\nu$, които ще използваме в числената си имплементация \cite{James_2015}.

\subsection{Ковариантна формулировка на уравнението за лъчист пренос}

Стандартната форма на уравнението за лъчист пренос през среда с функция на излъчването $j_\nu$ и на поглъщането $\alpha_\nu$, използвано в астрофизиката е \cite{Rad_processes}:

\begin{equation}
	\frac{dI_\nu}{ds} = j_\nu - \alpha_\nu I_\nu,
\end{equation}
където $s$ e дължината по траекторията на лъча и величината $I_\nu$ се нарича специфичен интензитет на монохроматичен сноп от фотони и се дефинира като количеството енергия, отнесена от снопа за единица време, от единица площ, в единица пространствен ъгъл, на единица честота. Изразено математически това е:
\begin{equation}
	I_\nu = \frac{dE}{dAdtd\nu d\Omega}.
\end{equation}
Недостатъкът на уравнение (2.21) е, че то не е инвариантно спрямо трансформации на Лоренц (понеже дължините и честотите не са инвариантни). За да го приведем в ковариантна форма, нека разгледаме теоремата на Лиувил. Тя гласи, че плътността във фазовото пространство, заемано от ансамбъл частици, подчиняващи се на уравненията на Хамилтън, е константа по интегралните криви на системата \cite{Misner1973}. Изразено математически това е:
\begin{equation}
	\frac{dN}{d\mathcal{V}} = \text{const}.
\end{equation}
Тук (за яснота, връщайки стойността на физичната константа $c$) $d\mathcal{V} = d^3x d^3p = \left[A cdt\right] p^2dpd\Omega$. В частния случай на фотони имаме $|p| = E / c$, oт което следва, че:
\begin{equation}
	\frac{dN}{d\mathcal{V}} = \frac{c^2dN}{E^2dAdtdEd\Omega} = \text{const}
\end{equation}
Правейки наблюдението, че ако разглеждаме един светлинен лъч като монохроматичен сноп от $N$ фотона, всеки от които има енергия $E$, можем да запишем отнесената от снопа енергия като $dE = EdN$. Замествайки това в уравнение (2.22) и сравнявайки с (2.24) имаме следното:
\begin{equation}
	\frac{dN}{d\mathcal{V}} = \frac{c^2}{h}\frac{I_\nu}{E^3} = \frac{c^2}{h^4}\frac{I_\nu}{\nu^3} = \text{const}
\end{equation}
От което следва, че инвариантният спрямо трансформации на Лоренц специфичен интензитет е:
\begin{equation}
	\mathcal{I}_\nu := \frac{I_\nu}{\nu^3}
\end{equation}
Ако въведем афинния параметър $\lambda$, дефиниран посредством проекцията на вълновия вектор $k^\mu$ върху скоростта на излъчващата среда $u_\mu$:
\begin{equation}
	\frac{ds}{d\lambda} = -u_\mu k^\mu = \nu
\end{equation}
Тогава можем да запишем уравнението за лъчист пренос (2.21) в ковариантна форма:
\begin{equation}
	\frac{d}{d\lambda}\left(\frac{I_\nu}{\nu^3}\right) = \frac{j_\nu}{\nu^2} - (\nu\alpha_\nu)\frac{I_\nu}{\nu^3}
\end{equation}
Ако дефинираме Лоренц инвариантните функции на излъчване $\mathcal{J}_\nu = \frac{j_\nu}{\nu^2}$ и поглъщане $\kappa_\nu = \nu\alpha_\nu$ можем да запишем уравнението за лъчист пренос като:
\begin{equation}
	\frac{d\mathcal{I}_\nu}{d\lambda} = \mathcal{J}_\nu - \kappa_\nu\mathcal{I}_\nu
\end{equation}
\subsubsection{Обобщение на уравнението за лъчист пренос за поляризиран сноп фотони}

Както видяхме от уравнение (\ref{a_orthog_k}), направлението на трептене на електромагнитната вълна е ортогонално на вълновия вектор. Тогава можем да дефинираме локална равнина, ортогонална на $k^\mu$, в която да описваме направлението на електромагнитните трептения. В тази равнина можем да фиксираме локална отправна система (наречена Стоксов базис \cite{Bronzwaer2020}\cite{Ipole_Code}, и дефинирана в допълнение \emph{Б}) и запишем двете степени на свобода на вектора на поляризация като \footnote{В нерелативистката литература, векторът $f^{(a)}$ може да се срещне в двумерна форма, наричан вектор на Джоунс.}:
\begin{equation}
f^{(a)} = \begin{pmatrix}
			0 \\
			E_{x} e^{i\phi_{x}} \\
			E_{y} e^{i\phi_{y}} \\
			0
	\end{pmatrix},
\end{equation}
където компонентите имат смисъл на нормирана проекцията на електричното поле върху Стоксовият базис. Представен по този начин $f^{(a)}$ има два недостатъка:\\

\noindent\textbf{1)} Той носи информацията за цялостната фаза на лъча, което го прави неудобен от наблюдателна гледна точка\footnote{Освен фазата, неудобство идва от това, че измерваемите величини при наблюденията са интензитети $\propto E_i^2$. Следователно би било по-удобно да се кодира поляризационното състояние чрез квадрата на елементите на $f^{(a)}$.}.\\

\noindent\textbf{2)} Описва единствено \emph{изцяло поляризирани} състояния, докато в реалните астрофизични процеси лъчението е само частично поляризирано.\\

\noindent С оглед на това се предпочита въвеждането на т.н. \emph{параметри на Стокс}, които кодират само относителната фаза между компонентите на $f^{(a)}$, и позволяват описването на частично поляризирани снопове, чрез въвеждане на величината $\mathcal{I}_{pol, \nu}$ представляваща \emph{специфичния интензитет на поляризираната компонента на лъчението}. Връзката между двете е следната \cite{Bronzwaer2020}:\\

\begin{equation}
	\mathcal{S}_\nu = \begin{pmatrix}
					\mathcal{I}_\nu \\
					\mathcal{Q}_\nu \\
					\mathcal{U}_\nu \\
					\mathcal{V}_\nu
				\end{pmatrix} = 
				\begin{pmatrix}
					\mathcal{I}_\nu \\
					\mathcal{I}_{pol, \nu} \left(f^{(1)} f^{(1)*} - f^{(2)} f^{(2)*} \right) \\
					\mathcal{I}_{pol, \nu} \left(f^{(1)} f^{(2)*} + f^{(2)} f^{(1)*} \right) \\
					i\mathcal{I}_{pol, \nu} \left(f^{(2)} f^{(1)*} - f^{(1)} f^{(2)*} \right)	
				\end{pmatrix} 
\end{equation}\\

\noindent Величината $\mathcal{I}_\nu$ има същия смисъл като в уравнение (2.29) - специфичен интензитет на \emph{пълното лъчение}. Компонентите $\mathcal{Q}_\nu$ и $\mathcal{U}_\nu$ кодират специфичния интензитет на линейно поляризирани компоненти, докато $\mathcal{V}_\nu$ - на кръгово поляризираните. Важно е да се отбележат две неща:\\\newline

\noindent\textbf{1)} Докато пълният специфичен интензитет удовлетворява условието $\mathcal{I}_\nu \ge 0$, знакът на останалите компоненти носи съществена информация.\\\newline

\noindent\textbf{2)} Величината $\mathcal{I}_{pol, \nu}$ се дефинира като $\mathcal{I}_{pol, \nu} = \sqrt{\mathcal{Q}_\nu^2 + \mathcal{U}_\nu^2 + \mathcal{V}_\nu^2}$ и удовлетворява условието $\mathcal{I}_{pol,\nu}\le\mathcal{I}_\nu$.\\\newline

\noindent С това обобщението на уравнение (2.29) се свежда до заместването $\mathcal{I}_\nu \rightarrow \mathcal{S}_\nu$, въвеждане на функции на излъчване $\{\mathcal{J}_\mathcal{I,\nu}, \,\mathcal{J}_\mathcal{Q,\nu},\, \mathcal{J}_\mathcal{U,\nu},\, \mathcal{J}_\mathcal{V,\nu}\}$ и поглъщане $\{\kappa_\mathcal{I,\nu}, \,\kappa_\mathcal{Q,\nu},\, \kappa_\mathcal{U,\nu},\, \kappa_\mathcal{V,\nu}\}$ за всяка компонента от вектора на Стокс и също така въвеждане на функциите на Фарадей$^3$ $\{\rho_\mathcal{Q,\nu},\, \rho_\mathcal{U,\nu},\, \rho_\mathcal{V,\nu}\}$. Така ковариантното уравнение за поляризиран лъчист пренос може да се запише като:

\begin{equation}
	\frac{d}{d\lambda} \begin{pmatrix}
							\mathcal{I}_\nu\\
							\mathcal{Q}_\nu\\
							\mathcal{U}_\nu\\
							\mathcal{V}_\nu
					     \end{pmatrix} = 
					     \begin{pmatrix}
					     	\mathcal{J}_\mathcal{I,\nu}\\
					     	\mathcal{J}_\mathcal{Q,\nu}\\
					     	\mathcal{J}_\mathcal{U,\nu}\\
					     	\mathcal{J}_\mathcal{V,\nu}
					     \end{pmatrix}
					     -	\begin{pmatrix*}[r]
					     	\kappa_\mathcal{I,\nu} & \kappa_\mathcal{Q,\nu} & \kappa_\mathcal{U,\nu} & \kappa_\mathcal{V,\nu}\\
					        \kappa_\mathcal{Q,\nu}& \kappa_\mathcal{I,\nu}& -\rho_\mathcal{U,\nu}& \rho_\mathcal{V,\nu}\\     	
					     	\kappa_\mathcal{U,\nu}& -\rho_\mathcal{V,\nu}& \kappa_\mathcal{I,\nu}& \rho_\mathcal{Q,\nu}\\	  
					     	 \kappa_\mathcal{V,\nu}& \rho_\mathcal{U,\nu}& -\rho_\mathcal{Q,\nu}& \kappa_\mathcal{I,\nu}\\
					     	\end{pmatrix*}
					     	\begin{pmatrix}
					     		\mathcal{I}_\nu\\
					     		\mathcal{Q}_\nu\\
					     		\mathcal{U}_\nu\\
					     		\mathcal{V}_\nu
					     	\end{pmatrix}
\end{equation}

\noindent Точната природа на функциите $\{\mathcal{J},\,\mathcal{\kappa},\,\mathcal{\rho}\}$ ще бъде обсъдена в глава \emph{3}, докато извеждането на някой от тях - в допълнение \emph{А}. В поляризирания случай, обаче, уравнение (2.32) трябва да бъде зацепено към (2.14) и те да се решат като една система. В общия случай това се извършва числено, при което е нужна не само трансформацията (26), но и обратната ѝ $\mathcal{S}_\nu\rightarrow f^{(a)}$. Тя обаче \emph{не е еднозначна} (поради това, че параметрите на Стокс "губят"$\,$ цялостната фаза на лъча). За щастие тази фаза не е физически значима за наблюдателя, и не оказва влияние върху динамичните уравнения. Следователно имаме свободата да изберем за нея произволна стойност. Стандартен избор е при извършване на прехода $\mathcal{S}_\nu\rightarrow f^{(a)}$, да наложим $\text{Im}{f^{(1)}} = 0$. Тогава тази трансформация се задава с \cite{Bronzwaer2020}:

\begin{equation}
	f^{(1)} = \sqrt{\frac{1 + \tilde{\mathcal{Q}}}{2}}, \quad f^{(2)} = \begin{cases}
		1,\quad \text{ако}\,\,f^{(1)} = 0 \\
		\frac{\tilde{\mathcal{U}} - i\tilde{\mathcal{V}}}{2f^{(1)}}
	\end{cases},
\end{equation}
където $\tilde{\mathcal{Q}} = \frac{\mathcal{Q}}{\mathcal{I}_{pol}}$, $\tilde{\mathcal{U}} = \frac{\mathcal{U}}{\mathcal{I}_{pol}}$, $\tilde{\mathcal{V}} = \frac{\mathcal{V}}{\mathcal{I}_{pol}}$.\\
\lfoot{
	\noindent\makebox[\linewidth]{\rule{\textwidth}{0.4pt}}
	$^3$ Функциите на Фарадей представляват реакцията на средата към преминаващото през нея лъчение и отговарят за смесването между различните компоненти на вектора на Стокс.
}

\noindent За да използваме уравнение (2.32) обаче, освен фиксиране на природата на излъчването ни е нужен и цялостен модел на излъчващата среда. Това включва познаване на плътността, химичния състав, температурата, скоростите и електромагнитните полета във всяка точка от средата. Имайки предвид целите поставени в глава \emph{1}, избраният от нас модел на средата неизбежно трябва да бъде информиран от съвременните VLBI наблюдения на обектите M87* и Sgr A*. В следващата глава ще представим основните резултати на колаборацията EHT, с помощта на които ще характеризираме излъчващата среда.
\lfoot{}

